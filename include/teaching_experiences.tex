\sectionTitle{Teaching Experiences}{\faShareAlt}

\vspace{0.5em}

\begin{itemize}
  \item OpenFOAM screencasts (in Persian) --
    {Sep 2022, \textsc{Babol Noshirvani University of Technology}}\vspace{0.5em}

    \hfill\begin{minipage}{\dimexpr\textwidth-0.92cm}
      These screencasts provide an introduction to developing custom OpenFOAM code in a Linux command-line environment.
      % The main purpose of the screencast was to prepare new students in our lab to modify and compile the source code of their solvers and libraries.
      % The first half of the screencasts covers working in the command-line interface (CLI) and familiarizing students with tools such as \texttt{find} and \texttt{grep}, as well as concepts like piping and redirection. The second half focuses on understanding the structure of the OpenFOAM source code, best practices for developing custom code (e.g., where to place the code and how to name solvers etc.), as well as code compilation.
    \end{minipage}\vspace{0.5em}

    \begin{itemize}
      \setlength\itemsep{0.3em}
      \item \link{https://www.youtube.com/playlist?list=PLdtuHsHJY9eicHEiF-xV6Iz4Jw5dQ3-ei}{{Command Line and Shell}}
      \item \link{https://www.youtube.com/playlist?list=PLdtuHsHJY9ejxkbkqBSpIHpGfhgte7cbY}{{OpenFOAM High Level Programming}}
    \end{itemize}
\end{itemize}
